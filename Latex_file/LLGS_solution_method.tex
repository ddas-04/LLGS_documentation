\documentclass[review]{elsarticle}

\usepackage{lineno,hyperref}
\usepackage{amsmath}
\usepackage{amsfonts}
\usepackage{amssymb}
\usepackage{subfigure}
\usepackage[margin=0.75in]{geometry}
\usepackage{multirow}
%\modulolinenumbers[5]
%\usepackage{lineno}

\journal{Arxiv}

%%%%%%%%%%%%%%%%%%%%%%%
%% Elsevier bibliography styles
%%%%%%%%%%%%%%%%%%%%%%%
%% To change the style, put a % in front of the second line of the current style and
%% remove the % from the second line of the style you would like to use.
%%%%%%%%%%%%%%%%%%%%%%%

%% Numbered
%\bibliographystyle{model1-num-names}

%% Numbered without titles
%\bibliographystyle{model1a-num-names}

%% Harvard
%\bibliographystyle{model2-names.bst}\biboptions{authoryear}

%% Vancouver numbered
%\usepackage{numcompress}\bibliographystyle{model3-num-names}

%% Vancouver name/year
%\usepackage{numcompress}\bibliographystyle{model4-names}\biboptions{authoryear}

%% APA style
%\bibliographystyle{model5-names}\biboptions{authoryear}

%% AMA style
%\usepackage{numcompress}\bibliographystyle{model6-num-names}

%% `Elsevier LaTeX' style
\bibliographystyle{elsarticle-num}
%%%%%%%%%%%%%%%%%%%%%%%

\begin{document}
%\linenumbers
\begin{frontmatter}

\title{Numerical Solution of Landau-Lifshitz-Gilbert-Slonczewski equation}
%\tnotetext[mytitlenote]{Fully documented templates are available in the elsarticle package on \href{http://www.ctan.org/tex-archive/macros/latex/contrib/elsarticle}{CTAN}.}

%% Group authors per affiliation:
\author{Debasis~Das}
\address{Department of Electrical and Computer Engineering, National University of Singapore}
%\fntext[myfootnote]{Since 1880.}

%% or include affiliations in footnotes:
%\author[mymainaddress,mysecondaryaddress]{Elsevier Inc}
%\ead[url]{www.elsevier.com}
%
%\author[mysecondaryaddress]{Debasis Das}
\cortext[mycorrespondingauthor]{Corresponding author\\ Debasis Das (eledd@nus.edu.sg)}
%\cortext[mycorrespondingauthor]{Corresponding author\\ BhaskaranMuralidharan (bm@ee.iitb.ac.in)}

%\ead{ddas@ee.iitb.ac.in}
%
%\address[mymainaddress]{1600 John F Kennedy Boulevard, Philadelphia}
%\address[mysecondaryaddress]{360 Park Avenue South, New York}
\begin{abstract}
	will be added later
\end{abstract}

\begin{keyword}
	MTJ, LLGS
\end{keyword}

\end{frontmatter}

%\linenumbers

\section{Introduction}
The basic equation that governs the motion of the nanomagnet is described by Landau-Lifshitz-Gilbert-Slonczewski(LLGS) equation. It is given by [eq. (8) of Ref. \cite{sengupta2017encoding}] 
\begin{equation}
	\frac{d\hat{m}}{dt}=-\frac{\gamma \mu_0}{1+\alpha^2}\left(\hat{m}\times \overrightarrow{H}_{eff}\right) -\frac{\gamma \mu_0 \alpha}{1+\alpha^2}\left(\hat{m}\times\hat{m}\times \overrightarrow{H}_{eff}\right) -\frac{1}{(1+\alpha^2)qN_s}\left(\hat{m}\times\hat{m}\times\overrightarrow{I}_s\right) +\frac{\alpha}{(1+\alpha^2)qN_s}\left(\hat{m}\times\overrightarrow{I}_s\right)
	\label{LLGS}
\end{equation}
Here $\gamma$ = 1.76$\times10^{11}$ rad/(S.T) is the Gyromagnetic ratio, $\mu_0 = 4\pi \times 10^{-7}$ T.m/A is the free space permeability and $\alpha$ is the damping constant. $\hat{m}$ is the unit magnetization vector whose dynamics we would describe by solving the LLGS equation. $\overrightarrow{H}_{eff}$ is the effective magnetic field which consists of uniaxial field($\overrightarrow{H}_{uni}$), demagnetization field($\overrightarrow{H}_{demag}$), and applied field($\overrightarrow{H}_{app}$). These fields gives the deterministic motion, but apart from that thermal noise can also influence the motion. This can be included by adding the thermal field($\overrightarrow{H}_{Therm}$) which is generated due to the thermal noise. So the effective field is given by 
\begin{equation}
	\overrightarrow{H}_{eff}=\overrightarrow{H}_{uni} + \overrightarrow{H}_{demag} + \overrightarrow{H}_{app} + \overrightarrow{H}_{Therm}
\end{equation}
$\overrightarrow{H}_{therm}$ is given by the following formula
\begin{equation}
	\overrightarrow{H}_{therm} = \sqrt{\frac{\alpha}{1+\alpha^2}\frac{2 K_B T_K}{\gamma M_s V \delta_t}}~\overrightarrow{G}_{0,1}
\end{equation}
Here, $K_B=1.38\times 10^{-23}~J/K$ is the Boltzmann constant, $T_K$ is the temperature, $M_s$ is the saturation magnetization, $V$ is the volume of the nanomagnet and $\delta_t$ is the simulation step. $\overrightarrow{G}_{0,1}$ is the Gaussian distribution with zero mean and unit standard deviation. \\
In Eq.(\ref{LLGS}), $q$ is the electronic charge, $N_s=\frac{M_s V}{\mu_B}$ is the number of free spins that are present in the free layer of the MTJ($\mu_B$ is the Bohr magneton). $\overrightarrow{I}_s$ in Eq.(\ref{LLGS}) denotes the spin current vector that enters into the free layer. $\mu_B$ is written as $\mu_B=\frac{\gamma \hbar}{2}$, so total number of free spins $N_s$ can be written as 
\begin{equation*}
	N_s=\frac{M_s V}{\mu_B}=\frac{2 M_s V}{\gamma \hbar}
\end{equation*}
Substituting this into Eq.(\ref{LLGS}), we get,
\begin{equation}
	\frac{d\hat{m}}{dt}=-\frac{\gamma \mu_0}{1+\alpha^2}\left(\hat{m}\times \overrightarrow{H}_{eff}\right) -\frac{\gamma \mu_0 \alpha}{1+\alpha^2}\left(\hat{m}\times\hat{m}\times \overrightarrow{H}_{eff}\right) -\frac{\gamma \hbar}{(1+\alpha^2)q 2 M_s V}\left(\hat{m}\times\hat{m}\times\overrightarrow{I}_s\right) +\frac{\alpha \gamma \hbar}{(1+\alpha^2)q 2 M_s V}\left(\hat{m}\times\overrightarrow{I}_s\right)
\end{equation}
This Eq. can further be simplified by considering $\overrightarrow{I}_s=|I_s|\hat{m}_p$, where $|I_s|$ denotes the magnitude of the spin current and $\hat{m}_p$ is the spin polarization direction.
\begin{equation}
	\frac{dm}{dt}=-\frac{\gamma \mu_0}{1+\alpha^2}\left(\hat{m}\times \overrightarrow{H}_{eff}\right) -\frac{\gamma \mu_0 \alpha}{1+\alpha^2}\left(\hat{m}\times\hat{m}\times \overrightarrow{H}_{eff}\right) -\frac{\gamma \hbar |I_s|}{(1+\alpha^2)2 q  M_s V}\left(\hat{m}\times\hat{m}\times\hat{m}_p\right) +\frac{\alpha \gamma \hbar |I_s|}{(1+\alpha^2)2 q M_s V}\left(\hat{m}\times\hat{m}_p\right)
	\label{LLGS_Ismag}
\end{equation}

The volume of the nanomagnet can be written as $V=A\times t_{FL}$, where $A$ is the cross-sectional area and the $t_{FL}$ is the thickness of the free layer MTJ. So, denoting $\frac{|I_s|}{A}=J_{MTJ}$, spin current density, we can modify Eq. (\ref{LLGS_Ismag}), such that,
\begin{equation}
\frac{dm}{dt}=-\frac{\gamma \mu_0}{1+\alpha^2}\left(\hat{m}\times \overrightarrow{H}_{eff}\right) -\frac{\gamma \mu_0 \alpha}{1+\alpha^2}\left(\hat{m}\times\hat{m}\times \overrightarrow{H}_{eff}\right) -\frac{\gamma \hbar J_{MTJ}}{(1+\alpha^2)2 q  M_s t_{FL}}\left(\hat{m}\times\hat{m}\times\hat{m}_p\right) +\frac{\alpha \gamma \hbar J_{MTJ}}{(1+\alpha^2)2 q M_s t_{FL}}\left(\hat{m}\times\hat{m}_p\right)
\label{LLGS_mp}
\end{equation}
Assume,
$\beta=\frac{\gamma \hbar J_{MTJ}}{2 q M_s t_{FL}}$ and substitute this in the above equation, we get,
\begin{equation}
	\frac{dm}{dt}=-\frac{\gamma \mu_0}{1+\alpha^2}\left(\hat{m}\times \overrightarrow{H}_{eff}\right) -\frac{\gamma \mu_0 \alpha}{1+\alpha^2}\left(\hat{m}\times\hat{m}\times \overrightarrow{H}_{eff}\right) -\frac{\beta}{1+\alpha^2}\left(\hat{m}\times\hat{m}\times\hat{m}_p\right) +\frac{\alpha \beta}{1+\alpha^2}\left(\hat{m}\times\hat{m}_p\right)
	\label{LLGS_final}
\end{equation}
Eq. (\ref{LLGS_final}) denotes the final LLGS equation that we need to solve for the magnetization dynamics.\\


\section{In-plane anisotropy}
For in-plane anisotropy suppose the easy axis is along the z-axis and the demagnetization field is along the y-axis, perpendicular to plane. So the associated fields are:\\
\underline{Uniaxial field}: $H_{uni}$=$\left[0,0,H_km_z\right]$, here $H_k=\frac{2 K_u}{\mu_0 M_s}$\\
\underline{Demagnetization field}: $H_{demag}$=$\left[0,-H_d m_y, 0\right]$, where $H_d=M_s$(for S.I. unit)\\
\underline{Critical current}: $I_{sc}$=$\frac{2 q \alpha}{\hbar}\left[\mu_0(H_k+0.5H_d)M_s V\right]$

\section{Perpendicular anisotropy}
For perpendicular anisotropy, both, uniaxial and the demagnetization field are perpendicular to plane but they oriented opposite to each other.\\
\underline{Uniaxial field}: $H_{uni}$=$\left[0,0,H_km_z\right]$\\%, here $H_k=\frac{2 K_u}{\mu_0 M_s}$\\
\underline{Demagnetization field}: $H_{demag}$=$\left[0,0,-H_d m_z\right]$\\ %, where $H_d=M_s$(for S.I. unit)\\
\underline{Critical current}: $I_{sc}$=$\frac{2 q \alpha}{\hbar}\left[\mu_0(H_k-H_d)M_s V\right]$










\section{References}

\bibliography{References}

\end{document}